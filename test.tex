\documentclass{article}
\usepackage{hyperref}
\usepackage{listings}
\usepackage{amstext}
\usepackage{alltt}
\usepackage{amsmath}
\usepackage[T1]{fontenc}
\setlength{\parskip}{\smallskipamount}
\setlength{\parindent}{0pt}
\usepackage{booktabs}
\begin{document}


From the relationship between $W^{+}$ and $W^{-}$: 
  \begin{align*}
 W^{+} = a \cdot d \cdot{W^{-}}^{s} + a \cdot (1-d) \cdot {\hat{W^{-}}} ^{s} 
 \end{align*}
 where $\hat{W^{-}}$ is the independent copy of $W^{-}$. Applying the double integral, the probability of $W^{+}$ is thus the following:
  \begin{align*}
 P( W^{+} > x ) & = P( d \cdot{W^{-}}^{s} + (1 - d) \cdot {\hat{W^{-}}} ^{s} > \frac{x}{a} ) \\
          & =  {(\frac{c_{1}}{c})}^{-\beta} +  {(\frac{c_{2}}{c})}^{-\beta} -  {(\frac{ c_{1}c_{2}   }{ c^{2} })}^{-\beta}
 \end{align*}
 
 where $c_{1}$ is the integral boundary value for $W^{-}$: $c_{1} = {\frac{1}{d}}^{\frac{1}{s}} \cdot {( \frac{x}{a} - (1-d) \cdot c^{s})}^{\frac{1}{s}}$; and $c_{2}$ for $\hat{W^{-}}$: 
 $c_{2} = {\frac{1}{1-d}}^{\frac{1}{s}} \cdot {( \frac{x}{a} - d \cdot c^{s})}^{\frac{1}{s}}$. Hence the power law distribution:
 \begin{align*}
\lim_{x \rightarrow +\infty } \frac{ {(\frac{c_{1}}{c})}^{-\beta} +  {(\frac{c_{2}}{c})}^{-\beta} -  {(\frac{ c_{1}c_{2}   }{ c^{2} })}^{-\beta}  }{x^{-\alpha}} = constant
\end{align*}
 \quad\\

To satisfy the power law requirement, we need:
  \begin{align*}
  s = \frac{\beta} {\alpha} 
 \end{align*}
 \quad\\
 By definition, the lower bound for $W^{+}$ is parameter b, which could also be shown in the form of a and c:
   \begin{align*}
  b = a \cdot d \cdot c^{s} + a \cdot (1-d) \cdot c^{s} = a \cdot c^{s}
 \end{align*}
 
 Additionally, as the mean of in-degree should be equal to that of out-degree, we have: 
  \begin{align*}
  E(W^{+}) = E(W^{-})  & \Rightarrow    a \cdot E({W^{-}}^{s}) = E({W^{-}}^{s})  \\
 & \Rightarrow    a \cdot \frac{\beta}{\beta - s} \cdot c^{s} = \frac{\beta}{\beta-1} \cdot c  
 \end{align*}
 Specifically in consideration of the case where s = 1, given that case in-degree and out-degree are supposed to be equal, thus identical $W^{+}$ and $W^{-}$, a should also be 1.
 
Notes: \\
\texttt{Samp\_Dist\_Corr}\\

\text{The eigenvector calculation is done by the power iteration method and has no guarantee of convergence. The iteration will stop after max\_iter iterations or an error tolerance of number\_of\_nodes(G)\*tol has been reached.}\\

\text{\#}

\begin{align*}
s &= \frac{\beta}{\alpha}\\
a\frac{\beta c^s}{\beta -s}&=\frac{\beta c}{\beta -1}\\
\Rightarrow 
 a &= 1 \text{ when } s=1 \\
a&=\frac{\beta - s}{a^{s-1}(\beta-1)}, \text{ otherwise}
\end{align*}

para 1 a long long word awd mesf mawdf mfesfmsef msfe sergmmggrdmmm rdgdrgd drg. \\

para 2  a long long word awd mesf mawdf mfesfmsef msfe sergmmggrdmmm rdgdrgd drg. \\
para3  a long long word awd mesf mawdf mfesfmsef msfe sergmmggrdmmm rdgdrgd drg. 
\par para4  a long long word awd mesf mawdf mfesfmsef msfe sergmmggrdmmm rdgdrgd drg. 


Leslie Lamport was the initial developer of \LaTeX, a document preparation system\cite{lam} based on \TeX.   \\
Let's see this webstie \href{https://en.wikibooks.org/wiki/LaTeX/Hyperlinks}{wiki page}
\href{https://networkx.github.io/documentation/networkx-1.11/reference/generated/networkx.algorithms.shortest_paths.generic.average_shortest_path_length.html?highlight=average%20shortest%20path%20length}{method description in doc}

\begin{enumerate}

\item 
\href{https://networkx.github.io/documentation/networkx-1.11/reference/generated/networkx.generators.degree_seq.directed_configuration_model.html?highlight=directed%20configuration%20model#networkx.generators.degree_seq.directed_configuration_model}{\text{directed\_configuration\_model}} \href{https://networkx.github.io/documentation/networkx-1.11/_modules/networkx/generators/degree_seq.html#directed_configuration_model}{$_{\text{[source code]}}$}

\begin{lstlisting}

directed_configuration_model(in_degree_sequence, out_degree_sequence, create_using=None, seed=None) 

\end{lstlisting}

Return a \text{directed\_random graph} with the given degree sequences.

The configuration model generates a random directed pseudograph (graph with parallel edges and self loops) by randomly assigning edges to match the given degree sequences.

To remove parallel edges:
\begin{lstlisting}
>>> D=nx.DiGraph(D)
\end{lstlisting}
To remove self loops:
\begin{lstlisting}
>>> D.remove_edges_from(D.selfloop_edges())
\end{lstlisting}

\end{enumerate}

\cite{networkx} \\
\cite{dcm}

From Table \ref{wikivote_stats}, we could see that...

\begin{table}[!hbp]
\centering
\caption{Wikipedia vote network statistics}\label{wikivote_stats}
\begin{tabular}{cc}

\toprule
Feature name & Value\\
\midrule
Nodes size & 7115\\
Edges & 103689\\
Expected degree & 14.57\\
In-out degree correlation & 0.317\\
Average clustering coefficient & 0.1409\\
Number of triangles & 608389\\
Diameter(longest shortest path) & 7\\

\bottomrule

\cline{1-2}
\hline
\end{tabular}

\end{table}

awff \\
\hypertarget{label1}{algorithm} aefaef \\


\newpage

we use the \hyperlink{label1}{algo} awfd\\


\begin{thebibliography}{999}

\bibitem{lam}
  Leslie Lamport,
  \emph{\LaTeX: A Document Preparation System}.
  Addison Wesley, Massachusetts,
  2nd Edition,
  \url{https://en.wikibooks.org/wiki/LaTeX/Hyperlinks}
  1994.
 

\bibitem{networkx} 
 \text{networkx package}
  \url{https://networkx.github.io/documentation/networkx-1.11/} 
  Released Jan 30, 2016
\bibitem{dcm} 
 \text{directed\_configuraiton\_model}
  \url{https://networkx.github.io/documentation/networkx-1.11/reference/generated/networkx.generators.degree_seq.directed_configuration_model.html?highlight=directed
  %20configuration%20model#networkx.generators.degree_seq.directed_configuration_model}
  
\end{thebibliography}

\end{document}